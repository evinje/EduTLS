%\documentclass[pdftex,10pt,b5paper,twoside]{book}
\documentclass[12pt,b5paper,titlepage]{report}
\usepackage[lmargin=25mm,rmargin=25mm,tmargin=27mm,bmargin=30mm]{geometry}

\usepackage{listings}
\usepackage{float}
\usepackage[Glenn]{fncychap}
\usepackage{fancyhdr}
\usepackage[margin=2.5cm]{geometry}
\usepackage{parskip}
\usepackage{appendix}
\usepackage{lmodern}
\usepackage{acronym}
\usepackage[pdftex]{graphicx}
\usepackage{url}
%\usepackage{lineno}
\usepackage{color}
\usepackage[pdfborder={0 0 0}, colorlinks=true,citecolor=blue, linkcolor=blue,urlcolor=blue]{hyperref}
\usepackage[utf8]{inputenc}
\definecolor{darkgray}{gray}{0.30}
\lstset{language=Java,captionpos=b,tabsize=3,frame=lines,keywordstyle=\color{blue},commentstyle=\color{darkgray},stringstyle=\color{red},numbers=left,numberstyle=\tiny,numbersep=5pt,breaklines=true,showstringspaces=false,basicstyle=\footnotesize,emph={label}}
\renewcommand\contentsname{Table of Contents}
\setcounter{secnumdepth}{3}
\setcounter{tocdepth}{3}
\setlength{\parskip}{10pt} 
\pagenumbering{roman}
\pagestyle{plain}

\newcommand\myclass[1]{\textsf{#1}}
\newcommand\myfunction[1]{\textit{#1}}
\newcommand\myobject[1]{\textit{#1}}
\newcommand\mypackage[1]{\textsf{#1}}

\begin{document}
%\linenumbers
\sloppy

\title{Educational implementation of SSL/TLS}
\author{Eivind Vinje}

% FRONT PAGE
\begin{titlepage}
\begin{center}
\textsc{NORWEGIAN UNIVERSITY OF SCIENCE AND TECHNOLOGY\\
FACULTY OF  INFORMATION TECHNOLOGY, MATHEMATICS AND ELECTRICAL ENGINEERING} \\
\vspace{0.5cm} 
% crop-et fra http://www.ntnu.no/infoavdelingen/selvhjelp/logoer/ntnu/NTNU_engelsk_RGB.png
\includegraphics[scale=0.5]{img/NTNU-logo} \\
\vspace{2.5cm} 
\rule{\linewidth}{0.5mm} \\[0.4cm]
{\huge \bfseries Educational implementation of SSL/TLS}\\[0.4cm] 
\rule{\linewidth}{0.5mm} \\[3.5cm]
\textsc{\Large Eivind Vinje}\\[1.4cm]
\textsc{\Large December 2010}\\
\end{center}
\end{titlepage}

% PROBLEM DESCIPTION
\begin{center}
\textsc{NORWEGIAN UNIVERSITY OF SCIENCE AND TECHNOLOGY\\
FACULTY OF  INFORMATION TECHNOLOGY, MATHEMATICS AND ELECTRICAL ENGINEERING} \\
\vspace{0.5cm} 
% crop-et fra http://www.ntnu.no/infoavdelingen/selvhjelp/logoer/ntnu/NTNU_engelsk_RGB.png
\includegraphics[scale=0.5]{img/NTNU-logo} \\

\vspace{1.0cm}
{\Huge{PROJECT ASSIGNMENT}}
\vspace{1.0cm}

\begin{tabular}{ p{4cm} p{11cm}}
Student's name:	& Eivind Vinje \\
Course: & TTM4531 Information Security, Specialization Project \\
Title: & Educational implementation of SSL/TLS \\
%\vspace{1cm}
Description: & \\
\end{tabular}
{\small{\begin{tabular}{p{15cm}}
\vspace{0.2cm}
SSL/TLS is the most common and widely used secure protocol in Internet. It is a package of more than 30 cryptographic primitives and protocols. For students studying information security it is of a crucial importance to have a good understanding of how the different parts are working. The aim of the project will be to develop an educational implementation of SSL/TLS that could be used when teaching information security. The basic idea is to create a simple protocol that encrypts communication between two hosts, implemented in Java, where versions of DH, RSA, AES and SHA-1 are designed and implemented accompanied by a graphical interface that will monitor what is happening in every moment of the work of the algorithm (protocol).
\end{tabular}  }}

\begin{tabular}{ p{4cm} p{11cm}}
Deadline: & 2010-12-20 \\
Submission date: & 2010-12-16 \\
Department: & Department of Telematics \\
Supervisor: & Prof. Danilo Gligoroski\\\\
\end{tabular}
\vspace{0.5cm}

Trondheim, \today 

\vspace{1cm}
\line(1,0){150} \\
Danilo Gligoroski
\addcontentsline{toc}{chapter}{Problem Description}
\end{center}